\documentclass[12pt, titlepage]{article}
\title{%
  \Large Final Year Project Specification\\
  \large Session 2017-2018}
\author{Marcello De Bernardi}
\date{20/10/2017}


\begin{document}
\maketitle
\tableofcontents


\section{Project aims}
% State the design, development or research challenge
% that the project aims to solve.
% WHAT?
The aim of the project is to explore how effectively neural networks
can learn cryptographic functions - in order to bootstrap an end-to-end
encrypted communication protocol that ensures confidentiality - by developing
an adversarial neural network model to be trained and evaluated.

The minimum project aim is to independently reproduce the findings
presented by Abadi and Andersen in their paper CITATION HERE, which
demonstrated that adversarially trained neural networks are
capable of learning "how to perform forms of encryption end decryption,
and also how to apply these operations selectively in order to meet
confidentiality goals."

Moreover, if possible, the project aims to move beyond the exploratory
work carried out by Abadi and Andersen, as well as further work by other
teams PASQUALE LINKS GO HERE, and explore ways in which neural networks
may be trained to learn more effectively in this domain, or to produce
more convincing results.


\section{Methodology}
% Describe the various steps that you intend to follow in
% order for you to achieve your project aims.
% HOW?
The project
% Methodology is the core concept of how the aim will be achieved
% tasks are the subcomponents of a methodology
The initial phase of the project will consist of background reading in
machine learning (neural networks) and cryptography (private key encryption),
as well as a review of the research literature on the subject. Further
learning will include familiarization with the Python programming language as
well as the TensorFlow API.

Initial implementation efforts will go towards building a simple adversarial
neural network system, as well as designing a quantitative evaluation method
for the nets' performance. The interim report and risk assessment will be
written based on the results obtained from this prototype.

The prototype will be folled by a larger implementation, using more complex
neural network architectures.

Once the implementation is largely finalized, the model will be trained on the
HPC cluster to generate result data. Refinements may be made to the model at
5. Write report


\section{Project Milestones}
% Indicate what measurable/tangible components you will
% produce as part of this project.  This may take the form
% of deliverable document(s) or developmental milestones
% such as a working piece of software/hardware.
% WHEN?
1. Specification
2. Initial implementation for testing
3. Interim report
4. Finalized implementation
5. Training data
6. Project report


\section{Required knowledge, skills, tools, and resources}
% Indicate as far as possible the skills that are required for you
% to undertake this project.  Also include any software, hardware
% or other tools or resources that you believe you will need.
The project requires background knowledge in the areas of machine learning
and cryptography, with a particular emphasis on generative adversarial
models and private-key encryption.

The main skills involved are Python programming (particularly in procedural
and object-oriented styles), as well as working with the TensorFlow API and
other supporting toolsets. In addition, an understanding of neural network
architecture will be required.

A variety of free and proprietary development tools will be used. The
implementation will be written in Python, and make heavy use of TensorFlow,
an open-source software library for machine learning. TensorFlow provides
a supporting tool, TensorBoard, which allows for visualization of TensorFlow
models. Most of the development will be done in Atom, a modular and highly
customizable text editor. One of the primary appeals of Atom lies in a
package called Hydrogen, which allows the user to interactively run
snippets of code while inspecting variables or generating visual graphs. This
is similar to the popular Jupyter Notebook.

The software to be developed is expected to be shorter than 500 lines of code,
so no major resources will be required during the development phase. The
training phase will require more computational power than is available on a
commodity PC or laptop, so an access request to the Queen Mary University of
London HPC will be made.


\section{Timeplan}
% This can be a GANTT chart submitted with this document or a list
% of tasks, milestones and deliverables with timings.



\end{document}
